\documentclass{article}
\title{La fonction Softmax}
\usepackage[utf8]{inputenc}
\usepackage{amsmath}

\begin{document}

\maketitle


La fonction softmax, est une généralisation de la fonction logistique. Elle convertit un vecteur de K nombres réels en une distribution de probabilités sur K choix.
Le cadre d’utilisation de softmax est celui du \textbf{problème de classification multi-classe}. Dans un tel problème, on demande en général que le modèle prédictif renvoie :
\begin{itemize}
    \item Soit \textbf{la classe prédite} (identifiée par un numéro par exemple).
    \item Soit \textbf{la liste des probabilités d’appartenance à chaque classe}. Le modèle renvoie alors \textbf{un vecteur de probabilité}, c’est-à-dire un vecteur de nombres entre 0 et 1, dont la somme vaut 1.
\end{itemize}

Si le modèle est un réseau de neurones, alors on est obligé d’être dans le deuxième cas, car c’est un modèle qui doit être dérivable si l’on veut utiliser la méthode du gradient.

La fonction sigmoïde, qui est l’activation la plus standard, ne renvoie pas de vecteur de probabilités, car il n’est pas garanti que la somme des valeurs valent 1. On pourrait quand même utiliser cette activation, mais dans ce cas il y a les inconvénients suivants :
\begin{itemize}
    \item L’utilisateur du réseau doit à la main normaliser les valeurs pour obtenir un vecteur de probabilité.
    \item Lors de l’apprentissage, le réseau doit lui-même apprendre que les valeurs cibles sont des vecteurs de probabilités.
    \item Certaines métriques nécessitent que la prédiction donnée par le réseau soit un vecteur de probabilité, comme par exemple l’entropie croisée (ce n’est pas le cas pour la métrique MeanSquareError).
\end{itemize}

Ainsi, la fonction d’activation softmax sert à \textbf{forcer le réseau de neurones à renvoyer des vecteurs de probabilités}.

\section*{Définition de la fonction softmax :}

Le principe de la fonction softmax est simple : on part d'un vecteur $(x_0,x_1,...)$ avec les coefficients quelconques, et on veut obtenir un vecteur avec des coefficients positifs de somme 1.

Il suffit alors :
\begin{itemize}
    \item de remplacer chaque $x_i$ par $\exp(x_i)$ pour obtenir un nombre positif.
    \item puis de diviser par la somme des coefficients pour rendre une probabilité.
\end{itemize}

Ce qui donne comme formule $\text{softmax}(x_0,x_1,...) = (y_0,y_1,...)$ où les $y_i$ valent :
\[y_i = \frac{\exp(x_i)}{\sum_{p \geq 0} \exp(x_p)}\]

Exemple :

$x = (x_1,x_2,x_3) = (1,4,-2)$

Alors pour calculer $\text{softmax}(x)$ il faut calculer :
\[\exp(x_1) = \exp(1) = 2.71...\]
\[\exp(x_2) = \exp(2) = 54.59...\]
\[\exp(x_3) = \exp(3) = 0.13...\]

Ensuite on divise chacun de ces nombres par la somme $s = \exp(x_1) + \exp(x_2) + \exp(x_3) = 74.81...$ ce qui donne :
\[\text{softmax}(x)_1 = \frac{\exp(x_1)}{s}\]
\[\text{softmax}(x_2)_2 = \frac{\exp(x_2)}{s}\]
\[\text{softmax}(x_3)_3 = \frac{\exp(x_3)}{s}\]

Ainsi $\text{softmax}(x) = (0.047...,0.95,0.002...)$ et on peut vérifier que la somme des coefficients fait bien 1.

\section*{Lien avec la fonction argmax :}

Mais alors d'où vient le nom softmax ? Il vient du fait que c'est une approximation de la fonction $argmax$ en une fonction lisse $C^{\infty}$.

La version encodée One hot de $argmax$ renvoie un vecteur de probabilité de type Dirac, c'est-à-dire une valeur 1 et que des 0 ailleurs. Par exemple sur le vecteur $x = (1,4,-2)$, la version encodée One Hot de son argmax est le vecteur (0,1,0) où la position du 1 est la position du plus grand coefficient de $x$. L'inconvénient de cette fonction n'est pas dérivable... On introduit alors cette nouvelle fonction pour pallier à ça. 

En effet, on peut voir que $\text{softmax}(x) = (0.047...,0.95,0.002...)$ est une assez bonne approximation de (0,1,0), ainsi softmax est une approximation dérivable de la version encodée One hot de $argmax$, d'où le nom "soft".

\section*{Proposition :}

La fonction $\text{softmax}$ $S$ est différentiable et $\frac{\partial S_i}{\partial x_j} = S_i(\delta_{i,j} - S_j)$ pour tout $i, j \in [\![1; n]\!]$, où $\delta_{i,j}$ est la fonction delta de Kronecker définie par $\delta_{i,j} = 1$ si $i = j$ et $0$ sinon. \\

\textit{Preuve :} 

\begin{proof}
On considère la fonction $\text{softmax}$ définie par :
\[S : \mathbb{R}^n \rightarrow [0,1]^n\]
\[(x_1,x_2,...,x_n) \rightarrow (S_1(x_1,...,x_n),...,S_n(x_1,...,x_n))\]

Ce que nous cherchons ce sont les dérivées de la $i$-ème sortie par rapport à la $j$-ème entrée, c'est-à-dire $\frac{\partial S_i}{\partial x_j}$.

Et on a :

\[
\frac{\partial S_i}{\partial x_j}(x_1, \ldots, x_n) = \frac{\partial}{\partial x_j}\left(\frac{e^{x_i}}{\sum_{k=1}^{n} e^{x_k}}\right)
\]

Si $i=j$ :
\[
\begin{aligned}
    \frac{\partial}{\partial x_j}\left(\frac{e^{x_i}}{\sum_{k=1}^{n} e^{x_k}}\right) &= \frac{e^{x_i}(\sum_{k=1}^{n}e^{x_k}) -e^{x_i}e^{x_j}}{(\sum_{k=1}^{n} e^{x_k})^2} \\
    &= e^{x_i}\frac{\sum_{k=1}^{n}e^{x_k}}{(\sum_{k=1}^{n}e^{x_k})^2} - \left(\frac{e^{x_i}}{\sum_{k=1}^{n}e^{x_k}}\right)^2 \\
    &= S_i(1 - S_i).
\end{aligned}
\]

Si $i \neq j$ :
\[
\begin{aligned}
    \frac{\partial}{\partial x_j}\left(\frac{e^{x_i}}{\sum_{k=1}^{n} e^{x_k}}\right) &= \frac{0 - e^{x_j}e^{x_i}}{(\sum_{k=1}^{n}e^{x_k})^2} \\
    &= -S_iS_j.
\end{aligned}
\]

Finalement, on a :
\[
\frac{\partial S_i}{\partial x_j} (x_1, \ldots, x_n) =
\begin{cases}
    S_i(1-S_j) & \text{si } i = j \\
    -S_iS_j & \text{si } i \neq j
\end{cases} = S_i(\delta_{i,j} - S_j).
\]
\end{proof}


\section*{Interprétation probabiliste :}

Les propriétés de $\text{softmax}$ (toutes les valeurs de sorties sont dans $[0,1]$ et s'additionnent à 1) rendent la fonction adaptée à une interprétation probabiliste très utile dans l'apprentissage automatique, en particulier dans les tâches de classification multiclasse.

On peut interpréter $\text{softmax}$ comme la probabilité prédite pour la jème classe étant donné un vecteur d’échantillon x et un vecteur de pondération w d'où \\
\[
P(y=j | \mathbf{x}) = \frac{e^{\mathbf{x}^{\mathsf{T}}\mathbf{w}_j}}{\sum_{k=1}^{K} e^{\mathbf{x}^{\mathsf{T}}\mathbf{w}_k}}
\]

\end{document}
